\documentclass[11pt]{article}
\usepackage{hyperref}
\usepackage{enumitem,fullpage,amsmath,amsthm,multirow,array,graphicx}
\usepackage{amsfonts}
\usepackage{comment}
\usepackage[procnames]{listings}
\usepackage{color} % used for including code
\definecolor{keywords}{RGB}{255,0,90}
\definecolor{comments}{RGB}{0,0,113}
\definecolor{red}{RGB}{160,0,0}
\definecolor{green}{RGB}{0,150,0}
 
\lstset{language=Python, 
        basicstyle=\ttfamily\small, 
        keywordstyle=\color{keywords},
        commentstyle=\color{comments},
        stringstyle=\color{red},
        showstringspaces=false,
        identifierstyle=\color{green},
        procnamekeys={def,class}}

\newcommand{\points}[1]{\textbf{[#1 Points]}}
\newcommand{\extracredit}{\mbox{\textbf{[Extra Credit]}}}

\newcommand{\Pm}[1]{\mathbb{P}\left[#1 \right]}
\title{CS186 Balanced Bidding}
\author{Luis Perez \& Tiffany Wu}
\date{\today}

\begin{document}
\maketitle

\section*{Balanced Bidding Agent}
\begin{enumerate}
\item We use the truthful bidding agent as an example. The  implementation of the expected utility function is below:

  % expected utility code depends on pset5_code/nautilusbb.py
  \lstinputlisting[language=Python, firstline=44, lastline=44]{pset5_code/nautilusbb.py}
  \lstinputlisting[language=Python, firstline=52, lastline=64]{pset5_code/nautilusbb.py}

  followed by the implementation of the bid function:

  % bid code depends on pset5_code/nautilusbb.py
  \lstinputlisting[language=Python, firstline=78, lastline=78]{pset5_code/nautilusbb.py}
  \lstinputlisting[language=Python, firstline=89, lastline=108]{pset5_code/nautilusbb.py}
\item 
  \begin{enumerate}
  \item 
  \end{enumerate}
\end{enumerate}

\section*{Experiments with Revenue: GSP vs VCG Auctions}
\begin{enumerate}[resume]
\item 
  \begin{enumerate}
  \item 
  \end{enumerate}
\end{enumerate}

\section*{The Competition}
\begin{enumerate}[resume]
\item
  \begin{enumerate}
  \item 
  \end{enumerate}
\end{enumerate}
\end{document}
